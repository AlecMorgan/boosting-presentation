\section{Appendix: Generalizing the Algorithm}

\begin{frame}
Recall the boosting algorithm

\textbf{Algorithm:} Gradient Boosting to Minimize Sum of Squared Errors.\\~\\

\textbf{Inputs:} A data set $\{ x_i, y_i \}$.

\textbf{Returns:} A function $f$ such that $f(x_i) \approx y_i$, minimizing least squared errors.

\begin{itemize}
  \item Initialize $f_0(x) = \frac{1}{N} \sum_i y_i$.
  \item Iterate (parameter $k$) until satisfied: \begin{itemize}
    \item Create the working data set $W_k = \{ x_i, y_i - f_k(x_i) \}$.
    \item Fit a decision tree to $W_k$, minimizing least squares (though most anything would work here).  Call this tree $T_k$.
    \item Set $f_{k+1}(x) = f_{k}(x) + T_{k}(x)$. 
  \end{itemize}
  \item Return $f_{\text{max}}(x) = f_0(x) + T_1(x) + T_2(X) + \cdots + T_{\text{max}}(x)$.
\end{itemize}
\end{frame}
%
\begin{frame}

Remember how we got here. \\~\\

We are trying to minimize least squared loss $L(f, y) = \frac{1}{2}(y - f)^2$.\\~\\

The gradient is $L(f, y) = f - y$.
\end{frame}
%
\begin{frame}
So maybe this works...
\end{frame}
%
\begin{frame}
\begin{itemize}
  \item Initialize $f_0(x) = \frac{1}{N} \sum_i y_i$.
  \item Iterate (parameter $k$) until satisfied: \begin{itemize}
    \item Create the working data set $W_k = \{ x_i, - \nabla_f L(f(x_i), y) \}$.
    \item Fit a decision tree to $W_k$, minimizing least squares (though most anything would work here).  Call this tree $T_k$.
    \item Set $f_{k+1}(x) = f_{k}(x) + T_{k}(x)$. 
  \end{itemize}
  \item Return $f_{\text{max}}(x) = f_0(x) + T_1(x) + T_2(X) + \cdots + T_{\text{max}}(x)$.
\end{itemize}

Close, but not quite...
\end{frame}
%
\begin{frame}
\textbf{Improvement One}:\\~\\

We chose to initialize to the average value of $y_i$ because it is the constant that minimized least squares.

Now that we have a different loss function, there's probably a better choice.\\~\\
\end{frame}
%
\begin{frame}
\textbf{Improvement One}:\\~\\

Initialize $f_0(x)$ to the constant $\mu$ minimizing $\sum_i L(\mu, y_i)$.
\end{frame}
%
\begin{frame}
\textbf{Improvement Two}: (Really the same improvement as one)\\~\\

In our update step we added the fit tree directly $f_{k+1}(x) = f_{k}(x) + T_{k}(x)$.\\~\\

The predictions from a regression tree are the \textit{average of of the response in the terminal nodes}.\\~\\

This happened to be the minimizer of the loss function (least squares) for those sample laying in some terminal node.
\end{frame}
%
\begin{frame}
\textbf{Improvement Two}:\\~\\

Better not to directly add the value that minimizes the loss function we care about.\\~\\

For each $x$ in a terminal node $R$, set $f_{k+1}(x) = f_k(x) + \mu$, where $\mu$ is chosen to minimize the loss function over all those training samples that lie in $R$:

$$ \mu = \argmin_t \sum_{x_i \in R} L(y_i, f_k(x_i) + t) $$
\end{frame} 


%-------------------
\section{Appendix: Gradient Boosted Logistic Regression Derivation}

\begin{frame}
Let's follow through our derivation of boosting and see what we get.
\end{frame}
%
\begin{frame}
\textbf{Question One}: How to initialize $f_0$?\\~\\

For squared error minimization, we initialized to the average value of $y_i$.

\begin{center}
\textit{Wait, why did we do that?}
\end{center}

\end{frame}
%
\begin{frame}
The average of $y_i$ is the \textit{constant} $\mu$ that minimizes the sum of squared errors.\\~\\

\begin{center}
    $\frac{1}{N} \sum_i y_i$ is the constant that minimizes
\end{center}

$$ \sum_i \left( y_i - \mu \right)^2 = \sum_i L(\mu, y_i) $$
\end{frame}
%
\begin{frame}
Ok, so maybe we should do the same thing here?

$$ f_0 = \argmin_{\mu} \sum_i L(\mu, y_i) = \sum_i \left( y_i \mu + \log(1 + e^{\mu}) \right) $$

\textbf{Question}: What is the minimizer (hint, what happens if you put \textit{only} an intercept in logistic regression)?

\end{frame}
%
\begin{frame}
The minimizer is the \textit{sample log odds}:

$$ f_0 = log \left( \frac{\frac{1}{N}\sum_i y_i}{1 - \frac{1}{N}\sum_i y_i } \right) $$ 
\end{frame}
%
\begin{frame}
\textbf{Question Two}: What's the gradient?\\~\\

$$ \nabla_f L(f, y) = \nabla_f \left( y f + \log(1 + e^f) \right) = y + \frac{e^f}{1 + e^f} $$

That last term is familiar...

$$ \frac{e^f}{1 + e^f} = p$$

Where $p$ is the probability associated with the log-odds $f$.
\end{frame}

